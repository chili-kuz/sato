\documentclass[a4paper,12pt]{jarticle}
\usepackage[dvipdfmx]{graphicx}
\usepackage{amsmath}
\usepackage{subfigure}
\usepackage{comment}
\usepackage{array}

\usepackage{ascmac} %枠つき文章

 
\setlength{\hoffset}{0cm}
\setlength{\oddsidemargin}{-3mm}
\setlength{\evensidemargin}{-3cm}
\setlength{\marginparsep}{0cm}
\setlength{\marginparwidth}{0cm}
\setlength{\textheight}{24.7cm}
\setlength{\textwidth}{17cm}
\setlength{\topmargin}{-45pt}

\renewcommand{\baselinestretch}{1.6}
\renewcommand{\floatpagefraction}{1}
\renewcommand{\topfraction}{1}
\renewcommand{\bottomfraction}{1}
\renewcommand{\textfraction}{0}
\renewcommand{\labelenumi}{(\arabic{enumi})}
%\renewcommand{\figurename}{Fig.} %図をFig.にする


%図のキャプションからコロン:を消す
\makeatletter
\long\def\@makecaption#1#2{% #1=図表番号、#2=キャプション本文
\sbox\@tempboxa{#1. #2}
\ifdim \wd\@tempboxa >\hsize
#1 #2\par 
\else
\hb@xt@\hsize{\hfil\box\@tempboxa\hfil}
\fi}
\makeatother

%auhorを右寄せに
\makeatletter
  \def\@maketitle{%
  \newpage\null
  \vskip 2em%
  \begin{center}%
  \let\footnote\thanks
    {\LARGE \@title \par}%
    \vskip 0em%titleとauthorの間隔
  \end{center}% 追加
    \mbox{}\hfill%% 追加
    {\large
      \lineskip 0.5em%
      \begin{tabular}[t]{c}%
        \@author
      \end{tabular}\par}%
    \vskip 3em%authorとdocumentの間隔
  \begin{center}% 追加
    {\large \@date}%
  \end{center}%
  \par\vskip 0.5em}
\makeatother


\begin{document}
%
\title{\vspace{-30mm} 制御系構成特論~レポート課題3}
\author{\underline{機械知能工学専攻~~16344217~~津上~祐典}}
\date{}
%
\maketitle
%
\vspace{-25mm}
%\parindent = 0pt %すべての段落で字下げをしない
%

%%%%%%%%%%%%%%%%%%%%%%%%%%%%%%
%\section*{\fbox{問題}}
%%%%%%%%%%%%%%%%%%%%%%%%%%%%%%
\begin{itembox}[l]{\Large{課題1.}}
%
 \begin{equation}
P(s)=\frac{1}{s-1}
 \end{equation}
%
 のとき,また
%
\begin{eqnarray}
 N_r=N_l=\frac{1}{s+1} \\
 D_r=D_l=\frac{s-1}{s+1}
\end{eqnarray}
%
であるとき,ベズー等式の一般解を求めよ.
\end{itembox}

\vspace{-10mm}
%%%%%%%%%%%%%%%%%%%%%%%%%%%%%%
\section*{\fbox{解答}}
%%%%%%%%%%%%%%%%%%%%%%%%%%%%%%
べズーの等式は
%
\begin{eqnarray}
 X_rN_r+Y_rD_r=I_m\\
 N_lX_l+D_lY_l=I_p
\end{eqnarray}
%
で表される.$N_r,~N_l,~D_r,~D_l$を代入すると,
%
\begin{eqnarray}
 X_r\cdot\frac{1}{s+1}+Y_r\cdot\frac{s-1}{s+1}=1\\
 X_l\cdot\frac{1}{s+1}+Y_l\cdot\frac{s-1}{s+1}=1
\end{eqnarray}
%
となり,
%
\begin{eqnarray}
 X_r=X_l=2\\
 Y_r=Y_l=1
\end{eqnarray}
%
を得る.これよりベズー等式を満足する解
$\bar{X_r},\bar{Y_r},\bar{X_l},\bar{Y_l}$を
%
\begin{eqnarray}
 \bar{X_r}&=&\bar{X_l}=2=\frac{2s+4}{s+2} \\
 \bar{Y_r}&=&\bar{Y_l}=1=\frac{s+2}{s+2}
\end{eqnarray}
%
とおく.また,自由パラメータ$Q,R$を
%
\begin{eqnarray}
 Q=R=-\frac{s+1}{s+2}
\end{eqnarray}
%
とするとベズー等式の一般解は,それぞれ
%
\begin{eqnarray}
 X_r&=&\bar{X_r}+QD_l \nonumber\\
 &=&\frac{2s+4}{s+2}-\left(\frac{s+1}{s+2}\right)\frac{s-1}{s+1} \nonumber\\
 &=&\frac{s+5}{s+2}
\end{eqnarray}
%
%
\begin{eqnarray}
 Y_r&=&\bar{Y_r}-QN_l \nonumber\\
 &=&\frac{s+2}{s+2}+\left(\frac{s+1}{s+2}\right)\frac{1}{s+1} \nonumber\\
 &=&\frac{s+3}{s+2}
\end{eqnarray}
%
%
\begin{eqnarray}
 X_l&=&\bar{X_l}+D_rR \nonumber\\
 &=&\frac{2s+4}{s+2}+\frac{s-1}{s+1}\cdot\left(-\frac{s+1}{s+2}\right) \nonumber\\
 &=&\frac{s+5}{s+2}
\end{eqnarray}
%
%
\begin{eqnarray}
 Y_l&=&\bar{Y_l}-N_rR \nonumber\\
 &=&\frac{s+2}{s+2}+\frac{1}{s+1}\left(\frac{s+1}{s+2}\right) \nonumber\\
 &=&\frac{s+3}{s+2}
\end{eqnarray}
%
となる.
\newpage
%%%%%%%%%%%%%%%%%%%%%%%%%%%%%%
\begin{itembox}[l]{\Large{課題2.}}
%
 \begin{equation}
 P(s)=\frac{1}{s-1}=[1,1,1,0]
 \end{equation}
 %
 において,$H=F=2$としたときの2重既約分解表現を求めよ.
\end{itembox}

\vspace{-10mm}
%%%%%%%%%%%%%%%%%%%%%%%%%%%%%%
\section*{\fbox{解答}}
%%%%%%%%%%%%%%%%%%%%%%%%%%%%%%
$P(s)$をドイルの記法で表現すると,
%
\begin{equation}
 P(s)=\frac{1}{s-1}=
\left[
  \begin{array}{c|c}
   1 ~&~ 1  \\ \hline
   1 ~&~ 0   \\ 
  \end{array}
  \right]
=
\left[
  \begin{array}{c|c}
   A ~&~ B  \\ \hline
   C ~&~ D   \\ 
  \end{array}
  \right]
\end{equation}
%
となる.ここで
%
\begin{eqnarray}
 A_H&=&A-HC=1-2\cdot1=-1\\
 A_F&=&A-BF=1-1\cdot2=-1\\
 B_H&=&B-HD=1-2\cdot0=1\\
 C_F&=&C-DF=1-0\cdot2=1
\end{eqnarray}
%
であるから二重既約分解表現すると
%
\begin{eqnarray}
\left[
  \begin{array}{c|c}
   Y_r ~&~ X_r  \\ \hline
   -N_l &~ D_l   
  \end{array}
\right]
=
\left[
  \begin{array}{c|cc}
  A_H & B_H  & H  \\ \hline
  F   &  I_m & 0  \\
  -C  & -D   & I_p
  \end{array}
\right]
=
\left[
  \begin{array}{c|cc}
  -1  ~&~ 1 ~& 2  \\ \hline
  -2  ~&~ 1 ~& 0  \\
  -1  ~&~ 0 ~& 1
  \end{array}
  \right]
\\
\left[
  \begin{array}{c|c}
   D_r ~& -X_l  \\ \hline
   N_r ~& Y_l   
  \end{array}
\right]
=
\left[
  \begin{array}{c|cc}
  A_F & B    & H  \\ \hline
  -F  &  I_m & 0  \\
  C_F & -D   & I_p
  \end{array}
\right]
=
\left[
  \begin{array}{c|cc}
  -1  ~&~ 1 ~& 2  \\ \hline
  -2  ~&~ 1 ~& 0  \\
   1  ~&~ 0 ~& 1
  \end{array}
  \right]
\end{eqnarray}
となる.
%
\newpage
%%%%%%%%%%%%%%%%%%%%%%%%%%%%%%%%%%%%%%%%
 \begin{itembox}[l]{\Large{課題3.}}
%
  \begin{equation}
   P(s)=\frac{1}{s-1}=[1,1,1,0]=\left[
  \begin{array}{c|c}
   1 ~&~ 1  \\ \hline
   1 ~&~ 0   \\ 
  \end{array}
  \right]
  \end{equation}
%
において,正規化右・左既約分解表現を求めよ.  
 \end{itembox}
\vspace{-10mm}
%%%%%%%%%%%%%%%%%%%%%%%%%%%%%%
\section*{\fbox{解答}}
%%%%%%%%%%%%%%%%%%%%%%%%%%%%%%
はじめにリッカチ方程式を解くと
%
 \begin{eqnarray}
  X(A-BR^{-1}D^{\mathrm{T}}C)+(A-BR^{-1}D^{\mathrm{T}}C)^{\mathrm{T}}X-XBR^{-1}B^{\mathrm{T}}X+C^{\mathrm{T}}\tilde{R}^{-1}C=0\\
 X(1-1\cdot1\cdot0\cdot1)+(1-1\cdot1\cdot0\cdot1)X-X\cdot1\cdot1\cdot1\cdot
  X+1\cdot1\cdot1=0\\
  -X^2+2X+1=0
\end{eqnarray}
%
$X>0$より解は$X=1+\sqrt{2}$となる.また,$Y$に関するリッカチ方程式は
%
 \begin{eqnarray}
(A-BD^{\mathrm{T}}\tilde{R}^{-1}C)Y+Y(A-BD^{\mathrm{T}}\tilde{R}^{-1}C)^{\mathrm{T}}-YC^{\mathrm{T}}\tilde{R}^{-1}CY+BR^{-1}B^{\mathrm{T}}=0\\
 (1-1\cdot0\cdot1\cdot1)Y+Y(1-1\cdot0\cdot1\cdot1)-Y\cdot1\cdot1\cdot1\cdot
  Y+1\cdot1\cdot1=0\\
  -Y^2+2Y+1=0
\end{eqnarray}
%
となり,解は$Y>0$より,$Y=1+\sqrt{2}$となる.
%
得られた$X,Y$より,
%
\begin{eqnarray}
 F&=&R^{-1}(D^{\mathrm{T}}C+B^{\mathrm{T}}X)=1(0\cdot1+1\cdot X)=1+\sqrt{2}\\
 H&=&(BD^{\mathrm{T}}+YC^{\mathrm{T}})\tilde{R}^{-1}=(1\cdot0+Y\cdot1)=1+\sqrt{2} \\
 A_F&=&A-BF=1-1-\sqrt{2}=1-\sqrt{2}\\
 A_H&=&A-HC=1-1-\sqrt{2}=1-\sqrt{2}\\
 C_F&=&C+DF=1\\
 B_H&=&B+DH=1
\end{eqnarray}
%
が求まり正規化右,左規約分解表現はそれぞれ,
%
\begin{eqnarray}
 \left[
  \begin{array}{c}
   D_r  \\ 
   N_r   \\ 
  \end{array}
\right]
&=&
\left[
  \begin{array}{c|c}
   A_F & BR^{-1/2}  \\ \hline
  -F   & R^{-1/2}   \\
  C_F  & DR^{-1/2} \\
  \end{array}
\right]
=
\left[
  \begin{array}{c|c}
   1-\sqrt{2} &~ 1  \\ \hline
  -1-\sqrt{2} &~ 1   \\
  1           &~0 \\ 
  \end{array}
\right]\\
 \left[
  \begin{array}{c}
   D_l  \\ 
   N_l   \\ 
  \end{array}
\right]
&=&
\left[
  \begin{array}{c|cc}
   A_H             & B_H               &-H \\ \hline
  \tilde{R}^{-1/2}C& \tilde{R}^{-1/2}D~& \tilde{R}^{-1/2} \\ 
  \end{array}
  \right]
=
\left[
  \begin{array}{c|cc}
   1-\sqrt{2} &~ 1 ~&-1-\sqrt{2} \\ \hline
  1           &~ 0 ~& 1 \\ 
  \end{array}
  \right]
\end{eqnarray}
となる.
%
\end{document}
