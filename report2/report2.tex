\documentclass[a4paper,12pt]{jarticle}
\usepackage[dvipdfmx]{graphicx}
\usepackage{amsmath}
\usepackage{subfigure}
\usepackage{comment}
\usepackage{array}

\usepackage{ascmac} %枠つき文章

 
\setlength{\hoffset}{0cm}
\setlength{\oddsidemargin}{-3mm}
\setlength{\evensidemargin}{-3cm}
\setlength{\marginparsep}{0cm}
\setlength{\marginparwidth}{0cm}
\setlength{\textheight}{24.7cm}
\setlength{\textwidth}{17cm}
\setlength{\topmargin}{-45pt}

\renewcommand{\baselinestretch}{1.6}
\renewcommand{\floatpagefraction}{1}
\renewcommand{\topfraction}{1}
\renewcommand{\bottomfraction}{1}
\renewcommand{\textfraction}{0}
\renewcommand{\labelenumi}{(\arabic{enumi})}
%\renewcommand{\figurename}{Fig.} %図をFig.にする


%図のキャプションからコロン:を消す
\makeatletter
\long\def\@makecaption#1#2{% #1=図表番号、#2=キャプション本文
\sbox\@tempboxa{#1. #2}
\ifdim \wd\@tempboxa >\hsize
#1 #2\par 
\else
\hb@xt@\hsize{\hfil\box\@tempboxa\hfil}
\fi}
\makeatother

%auhorを右寄せに
\makeatletter
  \def\@maketitle{%
  \newpage\null
  \vskip 2em%
  \begin{center}%
  \let\footnote\thanks
    {\LARGE \@title \par}%
    \vskip 0em%titleとauthorの間隔
  \end{center}% 追加
    \mbox{}\hfill%% 追加
    {\large
      \lineskip 0.5em%
      \begin{tabular}[t]{c}%
        \@author
      \end{tabular}\par}%
    \vskip 3em%authorとdocumentの間隔
  \begin{center}% 追加
    {\large \@date}%
  \end{center}%
  \par\vskip 0.5em}
\makeatother


\begin{document}
%
\title{\vspace{-30mm} 制御系構成特論~レポート課題2}
\author{\underline{機械知能工学専攻~~16344217~~津上~祐典}}
\date{}
%
\maketitle
%
\vspace{-25mm}
%\parindent = 0pt %すべての段落で字下げをしない
%

%%%%%%%%%%%%%%%%%%%%%%%%%%%%%%
%\section*{\fbox{問題}}
%%%%%%%%%%%%%%%%%%%%%%%%%%%%%%
\begin{itembox}[l]{\Large{問題}}
問1.~以下のシステム方程式を相似変換し,伝達関数を導出せよ.
  \begin{eqnarray}
   \dot{x}(t)&=&Ax(t)+Bu(t) \\
   y(t)&=&Cx(t)+Du(t)
  \end{eqnarray}

問2.~伝達関数$G(s)$が
\begin{equation}
 G(s)=\frac{s-a}{(s-a)^2+b^2}
  =\left[
  \begin{array}{rc|r}
  a  & b & 1 \\
   -b & a & 0 \\ \hline
   1  & 0 & 0
  \end{array}
  \right]
\end{equation}
で表され,$x(0)=[0~-\frac{1}{b}]$のとき,入力$u(t)=e^{at}$に対する応答
$y(t)$を求めよ.
 
問3.
 \begin{equation}
 G_1=\frac{s-1}{s+1}~,~G_2=\frac{1}{s-1}
 \end{equation}
 の場合について,直列,並列結合を求め,結合後のシステムの可制御・可観測性
 を調べよ.また,それぞれにおいて,結合後の伝達関数を求めよ.さらに,
 \begin{equation}
  G_1'=G_2'=\frac{1}{s-1}  
 \end{equation}
 のときの系列結合と可制御・可観測性を調べよ.
\end{itembox}

\vspace{-10mm}
%%%%%%%%%%%%%%%%%%%%%%%%%%%%%%
\section*{\fbox{解答}}
%%%%%%%%%%%%%%%%%%%%%%%%%%%%%%
問1.
$z(t)=Tx(t)$(ただし,$T$は正則行列)とおくと与式はそれぞれ
\begin{eqnarray}
 T^{-1}\dot{z}(t)&=&AT^{-1} z(t)+Bu(t) \nonumber\\
 \iff \dot{z}(t)&=&TAT^{-1}z(t)+TBu(t)\\
 y(t)&=&CT^{-1}z(t)+Du(t)
\end{eqnarray}
となる.初期値を零としラプラス変換すると
\begin{eqnarray}
 sZ(s)&=&TAT^{-1}Z(s)+TBU(s) \nonumber\\
 Z(s)&=&(sI-TAT^{-1})^{-1}TBU(s)
\end{eqnarray}
また,
\begin{equation}
 Y(s)=CT^{-1}Z(s)+DU(s)
\end{equation}
となる.ただし,$Y(s)=\mathcal{L}[(t)]~,~U(s)=\mathcal{L}[u(t)]$である.
代入すると,
%
\begin{equation}
 Y(s)=\left\{CT^{-1}(sI-TAT^{-1})^{-1}TB+D\right\}U(s)
\end{equation}
%
となる.これより伝達関数$G(s)$は
%
\begin{eqnarray}
 G(s)&=&\frac{Y(s)}{U(s)} \nonumber\\
 &=&CT^{-1}(sI-TAT^{-1})^{-1}TB+D \nonumber\\
 &=&CT^{-1}\left\{T(sI-A)T^{-1}\right\}^{-1}TB+D \nonumber\\
 &=&CT^{-1}T(sI-A)^{-1}T^{-1}TB+D \nonumber\\
 &=&C(sI-A)^{-1}B+D
\end{eqnarray}
%
となる.

問2.


問3.




\end{document}
