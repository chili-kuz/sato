\documentclass[a4paper,12pt]{jarticle}
\usepackage[dvipdfmx]{graphicx}
\usepackage{amsmath}
\usepackage{subfigure}
\usepackage{comment}
\usepackage{array}

\setlength{\hoffset}{0cm}
\setlength{\oddsidemargin}{-3mm}
\setlength{\evensidemargin}{-3cm}
\setlength{\marginparsep}{0cm}
\setlength{\marginparwidth}{0cm}
\setlength{\textheight}{24.7cm}
\setlength{\textwidth}{17cm}
\setlength{\topmargin}{-45pt}

\renewcommand{\baselinestretch}{1.6}
\renewcommand{\floatpagefraction}{1}
\renewcommand{\topfraction}{1}
\renewcommand{\bottomfraction}{1}
\renewcommand{\textfraction}{0}
\renewcommand{\labelenumi}{(\arabic{enumi})}
%\renewcommand{\figurename}{Fig.} %図をFig.にする


%図のキャプションからコロン:を消す
\makeatletter
\long\def\@makecaption#1#2{% #1=図表番号、#2=キャプション本文
\sbox\@tempboxa{#1. #2}
\ifdim \wd\@tempboxa >\hsize
#1 #2\par 
\else
\hb@xt@\hsize{\hfil\box\@tempboxa\hfil}
\fi}
\makeatother

\begin{document}
%
\title{\vspace{-30mm} \fbox{\large{制御系構成特論~レポート課題~~~機械知能工学専攻~~16344217~~津上~祐典}}}
\date{}
%
\maketitle
%
\vspace{-30mm}
%\parindent = 0pt %すべての段落で字下げをしない
%
%%%%%%%%%%%%%%%%%%%%%%%%%%%%%%
\section*{\fbox{問題}}
%%%%%%%%%%%%%%%%%%%%%%%%%%%%%%
厳密モデル$\tilde{P}(s)$,ノミナルモデル$P_n(s)$が
%
\begin{eqnarray}
 \tilde{P}(s)&=&\frac{2}{(s+1)(s-2)} \\
 P_n(s)&=&\frac{1}{s+1}
\end{eqnarray}
%
で表されるとき,$\Delta N_l(s)$,$\Delta D_l(s)$,加法的不確かさ$\Delta_a(s)$,
乗法的不確かさ$\Delta_m(s)$を求めよ.
\vspace{-10mm}
%%%%%%%%%%%%%%%%%%%%%%%%%%%%%%
\section*{\fbox{解答}}
%%%%%%%%%%%%%%%%%%%%%%%%%%%%%%
加法的不確かさ$\Delta_a(s)$,厳密モデル$\tilde{P}(s)$およびノミナルモデ
ル$P_n(s)$は以下の関係式を満たす.
%
\begin{equation}
 \tilde{P}(s)=P_n(s)-\Delta_a(s)
\end{equation}
%
これを式変形すると加法的不確かさ$\Delta_a(s)$
%
\begin{eqnarray}
 \Delta_a(s) &=& \tilde{P}(s)-P_n(s) \nonumber \\
  &=& \frac{2}{(s+1)(s-2)}-\frac{1}{s+1} \nonumber \\
 &=& \frac{-s+4}{(s+1)(s-2)} 
\end{eqnarray}
%
を得る.
%
また,乗法的不確かさ$\Delta_m(s)$,厳密モデル$\tilde{P}(s)$およびノミナ
ルモデル$P_n(s)$は以下の関係式を満たす.
%
\begin{equation}
\tilde{P}(s) = (1+\Delta_m(s))P_n(s) 
\end{equation}
%
これを式変形すると乗法的不確かさ$\Delta_m(s)$
%
\begin{eqnarray}
 \Delta_m(s) &=& \tilde{P}(s)P_n^{-1}(s)-1 \\
 &=&\frac{2(s+1)}{(s+1)(s-2)}-1 \\
 &=&\frac{-s+4}{s-2}
\end{eqnarray}
%
を得る.

次に,$\Delta N_l(s)$,$\Delta D_l(s)$を求める.$D_l(s),N_l(s)$の分母は
$(s+4)$であるので,それぞれ
%
\begin{eqnarray}
 D_l(s)&=&\frac{d_1}{s+4} \\
 N_l(s)&=&\frac{n_1}{s+4}
\end{eqnarray}
%
とおく.すると
%
\begin{eqnarray}
 P_n(s) &=& D_l^{-1}(s)N_l(s) \\
 \frac{1}{s+1} &=& \frac{s+4}{d}\cdot \frac{n}{s+4} 
\end{eqnarray}
%
より,$d_1=s+1,n_1=1$となり
%
\begin{eqnarray}
 D_l(s) &=& \frac{s+1}{s+4} \\
 N_l(s) &=& \frac{1}{s+4}
\end{eqnarray}
%
が求まる.また,$\Delta N_l(s),\Delta D_l(s)$の分母は$(s+2)(s+3)$であ
るのでそれぞれ
%
\begin{eqnarray}
 \Delta N_l &=& \frac{n_2}{(s+2)(s+3)} \\
 \Delta D_l &=& \frac{d_2}{(s+2)(s+3)}
\end{eqnarray}
%
とおく.ここで,$\Delta N_l(s),\Delta D_l(s)$は
%
\begin{eqnarray}
 \Delta N_l(s) &=& \tilde{N}_l(s) -N_l(s) \label{equ:n} \\
 \Delta D_l(s) &=& \tilde{D}_l(s) -D_l(s) \label{equ:d}
\end{eqnarray}
%
で表される.(\ref{equ:n})式より,
%
\begin{eqnarray}
 \frac{n_2}{(s+2)(s+3)} &=& \tilde{N}_l(s) - \frac{1}{s+4} \nonumber \\
 \tilde{N}_l(s) &=& \frac{n_2(s+4)+(s+2)(s+3)}{(s+2)(s+3)(s+4)} 
\end{eqnarray}
%
を得る.同様に(\ref{equ:d})式より,
%
\begin{eqnarray}
 \frac{d_2}{(s+2)(s+3)} &=& \tilde{D}_l(s) - \frac{s+1}{s+4} \nonumber \\
 \tilde{D}_l(s) &=& \frac{d_2(s+4)+(s+1)(s+2)(s+3)}{(s+2)(s+3)(s+4)} 
\end{eqnarray}
%
を得る.$\tilde{P}(s),\tilde{D}_l(s),\tilde{N}_l(s)$の関係式より,
%
\begin{eqnarray}
 \tilde{P}(s) &=& \tilde{D}_l^{-1}(s)\tilde{N}_l(s) \\
 \frac{2}{(s+1)(s-2)} &=&
  \frac{(s+2)(s+3)(s+4)}{d_2(s+4)+(s+1)(s+2)(s+3)} \cdot
  \frac{n_2(s+4)+(s+2)(s+3)}{(s+2)(s+3)(s+4)} \nonumber\\
 &=&\frac{n_2(s+4)+(s+2)(s+3)}{d_2(s+4)+(s+1)(s+2)(s+3)}
\end{eqnarray}
%
となり,
%
\begin{eqnarray}
 n_2 (s+4) &=& 2-(s+2)(s+3) \\
 d_2 (s+4) &=& (s+1)(s-2)-(s+1)(s+2)(s+3)
\end{eqnarray}
%
を得る.
\end{document}
